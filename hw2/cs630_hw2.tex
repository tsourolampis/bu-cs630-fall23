\documentclass[11pt]{article}
\usepackage{amsmath,amssymb,amsthm}

\DeclareMathOperator*{\E}{\mathbb{E}}
\let\Pr\relax
\DeclareMathOperator*{\Pr}{\mathbb{P}}

\newcommand{\eps}{\varepsilon}
\newcommand{\inprod}[1]{\left\langle #1 \right\rangle}


\newcommand{\handout}[5]{
	\noindent
	\begin{center}
		\framebox{
			\vbox{
				\hbox to 5.78in { {\bf CS 630: Graduate Algorithms } \hfill #2 }
				\vspace{4mm}
				\hbox to 5.78in { {\Large \hfill #5  \hfill} }
				\vspace{2mm}
				\hbox to 5.78in { {\em #3 \hfill #4} }
			}
		}
	\end{center}
	\vspace*{4mm}
}


\topmargin 0pt
\advance \topmargin by -\headheight
\advance \topmargin by -\headsep
\textheight 8.9in
\oddsidemargin 0pt
\evensidemargin \oddsidemargin
\marginparwidth 0.5in
\textwidth 6.5in
\parindent 0in
\parskip 1.5ex


\newcommand{\lecture}[4]{\handout{#1}{#2}{#3}{Due to: #4}{#1}}



\usepackage[ruled,vlined,linesnumbered]{algorithm2e}
\usepackage{lmodern}        % Latin Modern family of fonts
\usepackage[T1]{fontenc}
\usepackage[latin9]{inputenc}
\usepackage{titlesec}
\usepackage{titling}
\usepackage{verbatim}
\usepackage{float}
\usepackage{amssymb,amsfonts,amsmath}
\usepackage{enumerate}
\usepackage{enumitem}
\usepackage{amsthm}
\usepackage{graphicx,wrapfig,lipsum}
\usepackage{algpseudocode}
\usepackage{color}
\usepackage{graphicx}
\usepackage{tikz}
\usepackage{amsthm}
\usepackage{hyperref}
\usepackage{caption}
\usepackage{subfigure}
\usepackage{microtype}



\bibliographystyle{plainurl}
\newfloat{Figure}{tbp}{loa}
\floatname{Figure}{Figure}
\setlength{\droptitle}{-10em}


\renewcommand\qedsymbol{$\blacksquare$}


%==============================================================================

% Babis  Macros.

%==============================================================================

\newcommand{\problem}[1]{\section{#1}}		% Problem.
\newcommand{\new}[1]{{\em #1\/}}		% New term (set in italics).
\newcommand{\set}[1]{\{#1\}}			% Set (as in \set{1,2,3})
\newcommand{\setof}[2]{\{\,{#1}|~{#2}\,\}}	% Set (as in \setof{x}{x > 0})
\newcommand{\C}{\mathbb{C}}	                % Complex numbers.
\newcommand{\N}{\mathbb{N}}                     % Positive integers.
\newcommand{\Q}{\mathbb{Q}}                     % Rationals.
\newcommand{\R}{\mathbb{R}}                     % Reals.
\newcommand{\Z}{\mathbb{Z}}  			  	 % Integers.
\newcommand{\LL}{\mathcal{L}}               

\newcommand{\compl}[1]{\overline{#1}}	


\newcommand{\pr}[1]{\text{\bf Pr}\normalfont\lbrack #1 \rbrack} %probability
\newcommand{\ex}[1]{\mathbb{E}\normalfont\lbrack #1 \rbrack}%expectation
\newcommand{\bpr}[1]{\text{\bf Pr}\normalfont \Big[#1 \Big]} %probability
\newcommand{\bex}[1]{\mathbb{E}\normalfont \Big[#1 \Big]}
\newcommand{\ang}[1]{\langle #1 \rangle }


\newcommand{\m}{\mathfrak{m}}
\newcommand{\p}{\mathfrak{p}} 
\newcommand{\PP}{\mathbb{P}} 
\newcommand{\G}{\mathcal{G}}
\newcommand{\tx}[1]{\text{#1}}
\newcommand{\ttx}[1]{\texttt{#1}}



\theoremstyle{theorem}
\newtheorem{theorem}{Theorem}

\theoremstyle{lemma}
\newtheorem{lemma}{Lemma}

\theoremstyle{corollary}
\newtheorem{corollary}{Corollary}


\theoremstyle{definition}
\newtheorem*{definition}{Definition}

\newtheorem{proposition}{Proposition}
\newtheorem{example}{Example}
\newtheorem{remark}{Remark}
\newtheorem{fact}{Fact}
\newtheorem{claim}{Claim}



\begin{document}

\lecture{Homework 2}{Released on 9/29}{Prof. Charalampos Tsourakakis}{Friday 5pm, 10/13}

\begin{center}
    
\section*{Instructions}
\framebox{%
	\begin{minipage}{0.9\linewidth}
		\begin{itemize}
			\item The homework is due on \underline{{\bf Friday 10/13 at 5pm ET}}.  
			\item No extension will be provided, unless for serious documented reasons.
			\item {\bf Despite having two weeks, better start early than late!}  
   \item Unless specified differently, the points are distributed equally among the sub-questions. 
			\item Study the material taught in class, and feel free to do so in small groups, but the solutions should be a product of your own work. 
			\item This  is not a multiple choice homework;   reasoning, and mathematical proofs are required before giving your final answer.  
   \item {\bf Sub-optimal algorithms and lack of proof of correctness will lead to a major deduction in points. }
   \item If you work with others or utilize any external tools and resources, please make sure to annotate your answers.
   \item Please submit your work through Gradescope. You can find the access code on Piazza. 
   
		\end{itemize}
\end{minipage}}

\end{center}


\section*{Exercise 1 [10 points]}


Given an integer array \texttt{A} of length \( n \), and two integers \( k \) and \( \ell \), find the number of pairs of indices \( i, j \) (where \( 1 \leq i, j \leq n \)) such that \( \texttt{A}[i] \leq \texttt{A}[j] \) and there are exactly \( k \) elements in the range \( [\texttt{A}[i], \texttt{A}[j]] \) that are divisible by \( \ell \)\footnote{For example, consider $n = 5, \texttt{A} = [44, 13, 5, 7, 16, 40], k = 1, \ell=3$. The pair $(6,1)$ should be counted since $\texttt{A}[6]=40,\texttt{A}[1]=44$ and there is a single ($k=1$) number  $x$ such that $40\leq x \leq 44$ that is divisible by 3 (i.e., $x=42$).}. Give an efficient algorithm and analyze its time and space complexity. 



\section*{Exercise 2 [10 points]}



The input comprises a real number \( x \) and a matrix \( A[1..n, 1..m] \) containing \( nm \) real numbers. It is given that each row \( A[i, 1..m] \) is sorted in ascending order from left (i.e., first column $j=1$) to right (i.e., last column $j=m$), and each column \( A[1..n, j] \) is sorted in ascending order from top (i.e., first row $i=1$) to bottom  (i.e., last row $j=n$). The objective is to ascertain whether the value \( x \) is present within matrix \( A \).  Give an efficient algorithm and analyze its time and space complexity. 



\section*{Exercise 3 [10 points]} 

Given a sequence of \( n \) distinct numbers \(  \texttt{A} = \langle a_1, a_2, \ldots, a_n \rangle \), we define a pair \( (a_i, a_j) \) as an inversion if \( i < j \) but \( a_i > a_j \). The total number of inversions in the sequence \( A \), denoted as \( I(A) \), is the count of pairs \( (a_i, a_j) \) that are inversions.   Design a divide and conquer algorithm to determine \( I(A) \) in \( \Theta(n \log n) \) time.


\section*{Exercise 4 [10 points]} 

In a theater, there are \(n\) front seats arranged in a straight line. Due to COVID-19 policies, individuals are required to maintain an empty seat's distance from others on both their left and right. As individuals arrive, they randomly select and occupy a seat.

 \begin{enumerate}
         \item Give tight upper and    lower bounds  on the number of occupied seats. Prove the tightness of your bounds. 
     \item Write a probabilistic recurrence that expresses the expected number of occupied seats. 
 \end{enumerate}

  \section*{Exercise 5 [10 points]} 
 
Consider a Monte Carlo algorithm \( A \) for a problem \( \Pi  \) whose expected running time is at most \( T(n) \) on any instance of size \( n \) and that produces a correct solution with probability \( \gamma(n) \). Suppose further that given a solution to \( n \), we can verify its correctness in time \( t(n) \). Show how to obtain a Las Vegas algorithm that always gives a correct answer to \( \Pi  \). What is the expected run time of your procedure?
 
\section*{Exercise 6 [5 points]} 

Suppose you are given a coin for which the probability of HEADS, say \( p \), is unknown. How can you use this coin to generate unbiased (i.e., \( \Pr[\text{HEADS}] = \Pr[\text{TAILS}] = 1/2 \)) coin-flips? Give a scheme for which the expected number of flips of the biased coin for extracting one unbiased coin-flip is no more than \( \frac{1}{p(1 - p)} \).

 \section*{Exercise  7 [10 points]} 

A large dining hall has 137 seats available for a special dinner event, and every seat has been reserved for the night. The guests arrive in a random order to the hall. However, the 17th guest to arrive is confused and can't remember the number of his assigned seat, so he chooses a seat at random and sits down. From then on, whenever a guest arrives and finds their seat already taken, they also choose a random seat to sit. What is the probability that the last guest to arrive finds their assigned seat empty?


 \section*{Exercise  8 [10 points]} 


An alternative analysis of the running time of randomized quicksort focuses on the expected running time of each individual recursive call to \texttt{RANDOMIZED-QUICKSORT}, rather than on the number of comparisons performed.

\begin{enumerate}
    \item Argue that, given an array of size $n$, the probability that any particular element is chosen as the pivot is $\frac{1}{n}$.  
    
   
    \item Show that 
    \[
    \mathbb{E}[T(n)] = \frac{2}{n} \sum_{q=2}^{n-1} \mathbb{E}[T(q)] + \Theta(n).   \tag{8.1}
    \]
    
    \item Show that
    \[
    \sum_{k=2}^{n-1} k \lg k \leq \frac{1}{2} n^2 \lg n - \frac{1}{8} n^2. \tag{8.2}
    \]
 
    \item Using the bound from equation (8.2), show that the recurrence in equation (8.1) has the solution $\mathbb{E}[T(n)] = \Theta(n \lg n)$.  
\end{enumerate}

  \section*{Exercise 9 [10 points]} 

Prove the lower bound of \( \left\lfloor \frac{3n}{2} \right\rfloor - 2 \) comparisons in the worst case to find both the maximum and minimum of \( n \) numbers.

  \section*{Exercise 10 [15 points + 10 bonus points]} 

\subsection*{Theory part [15 points]}

For $n$ distinct elements $x_1,x_2\dots x_n$ with positive weights $w_1,w_2\dots w_n$ such that $\sum_{i=1}^n w_i=1$, the weighted median is the element $x_k$ satisfying $\sum_{x_i<x_k}w_i<{1\over 2}$ and $\sum_{x_i>x_k}w_i\leq {1\over 2}$. For example, if the elements are $0.1,0.1,0.1,0.3,0.4$ and each element equals its weight (so $w_i=x_i$ for $1\leq i\leq 5$), then the median is $0.1$, but the weighted median is $0.3$.

\begin{enumerate}
    \item[(a)] Argue that the median of $x_1,x_2\dots x_n$ is the weighted median of the $x_i$ with weights $w_i=1/n$ for $i=1,2,\dots n$.
    \item[(b)] Show how to compute the weighted median of $n$ elements in $O(n\log n)$ worst-case time using sorting.
    \item[(c)] Show how to compute the weighted median in $\Theta(n)$ worst-case time using a linear-time median algorithm.
   \end{enumerate} 
    Define the distance between two elements $a,b$ as $d(a,b)=|a-b|$. Given points $x_1,x_2\dots x_n$ with positive weights $w_1,w_2\dots w_n$ as defined previously, we wish to find a point $x$ (not necessarily one of the input points) that minimizes the sum $\sum_{i=1}^n w_i\cdot d(x,x_i)$. 
\begin{enumerate}

    \item[(d)] Argue that the weighted median is a best solution.

    \item[(e)] We generalize the problem into 2-dimensions. We are given a set of $n$ points $\{(x_i,y_i)\}_{i=1,\ldots,n}$, and their corresponding positive weights $w_1,w_2\dots ,w_n$ such that $\sum_{i=1}^n w_i=1$. What is a best solution point $(x,y)$ that minimizes its weighted distance sum to all the $n$ points in the set? Here the weighted distance between points $(x,y)$ and $(x_i,y_i)$ is defined as $w_i\cdot(|x_i-x|+|y_i-y|)$.
\end{enumerate}

\subsection*{Coding part [10 bonus points]}

Implement an algorithm that finds the best solution you described in the last question (e) of the theory part. Achieve this by completing the \textit{best\_position} function in the Python file available at the Git repo \url{https://github.com/tsourolampis/bu-cs630-fall23}. 

\end{document}

